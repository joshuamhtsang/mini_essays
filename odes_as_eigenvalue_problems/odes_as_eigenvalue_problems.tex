\documentclass[a4paper,11pt,landscape]{article}
\usepackage{amsmath}
\usepackage[a4paper,margin=1in,landscape]{geometry}
\usepackage[T1]{fontenc}
\usepackage[utf8]{inputenc}
\usepackage{lmodern}
\usepackage{amsfonts}
\usepackage{amssymb}
\usepackage{amsthm}
\usepackage{graphicx}
\usepackage{color}
\usepackage{xcolor}
\usepackage{url}
\usepackage{textcomp}
\usepackage{parskip}

\title{Ordinary Differential Equations as Eigenvalue and Eigenvector Problems}
\author{Joshua Man Hong Tsang}
\date{\today}

\begin{document}

\maketitle
\tableofcontents

\begin{abstract}
\end{abstract}

\section{Solution of First-Order Linear Ordinary Differential Equations}

Since higher-order ordinary differential equations (ODEs) can be decomposed into a system of coupled first-order ODEs, it is instructive to cover the solution of first-order ODEs as a starting point. Consider a first-order ODE of the form:
\begin{equation} \label{eqn:exp_growth} 
    \dot{x} = Ax
\end{equation}
where $\dot{x}$ denotes the time-derivative of $x$, $A$ is a constant (homogenous) and an initial value $x(t=0)$ is specified.  A general workflow towards a solution, regardless of the DE's order, may involve the following steps:
\begin{enumerate}
  \item Suppose the ansatz $x(t)=e^{\lambda t}$
  \item Derive the characteristic polynomial
  \item Solve for the roots of the characteristic polynomial $\{\lambda_i\}$
  \item Write the solution as a linear combination of the ansatz $e^{\lambda_i t}$
  \item Deduce the value of the solution coefficients using the initial value/conditions
\end{enumerate}

\eqref{eqn:exp_growth} and

\subsection{The importance of the characteristic polynomial}


\section{Solution of Second-Order ODE - The Harmonic Oscillator}




\section{Decomposition of Higher-Order }

\end{document}