\documentclass[a4paper,11pt,landscape]{article}
\usepackage{amsmath}
\usepackage[a4paper,margin=1in,landscape]{geometry}
\usepackage[T1]{fontenc}
\usepackage[utf8]{inputenc}
\usepackage{lmodern}
\usepackage{amsfonts}
\usepackage{amssymb}
\usepackage{amsthm}
\usepackage{graphicx}
\usepackage{color}
\usepackage{xcolor}
\usepackage{url}
\usepackage{textcomp}
\usepackage{parskip}

\title{Ordinary Differential Equations as Eigenvalue and Eigenvector Problems}
\author{Joshua Man Hong Tsang}
\date{\today}

\begin{document}

\maketitle
\tableofcontents

\begin{abstract}
\end{abstract}

\section{Solution of First-Order Linear Ordinary Differential Equations}

Since higher-order ordinary differential equations (ODEs) can be decomposed into a system of coupled first-order ODEs, it is instructive to cover the solution of first-order ODEs as a starting point. Consider a first-order ODE of the form:
\begin{equation} \label{eqn:exp_growth} 
    \dot{x} = Ax
\end{equation}
where $\dot{x}$ denotes the time-derivative of $x$, $A$ is a constant (homogenous) and an initial value $x(t=0) = x_0$ is specified.  It is the exponetial growth/decay ODE.  A general workflow towards a solution, regardless of the DE's order, may involve the following steps:
\begin{enumerate}
  \item Suppose the ansatz $x(t)=e^{\lambda t}$
  \item Derive the characteristic polynomial
  \item Solve for the roots of the characteristic polynomial $\{\lambda_i\}$
  \item Write the solution as a linear combination of the ansatz $e^{\lambda_i t}$
  \item Deduce the value of the solution coefficients using the initial value/conditions
\end{enumerate}

In the case of \eqref{eqn:exp_growth} the characteristic polynomial is a linear expression:
\begin{equation} 
    \lambda = A
\end{equation}
and thus the solution is composed of one term:
\begin{equation} 
    x(t)=C_1 e^{A t}
\end{equation}
where the value of the coefficient $C_1$ can be deduced using the initial condition:
\begin{equation} 
    x_0 = C_1
\end{equation}
The solution of \eqref{eqn:exp_growth} is therefore:
\begin{equation} 
    x(t)= x_0 e^{A t}
\end{equation}
The solution states that the initial value $x_0$ will either growth or decay with the exponential factor $e^{A t}}$ based on whether the constant $A$ is positive or negative.




\section{Solution of Second-Order ODEs - The Harmonic Oscillator}

Consider the second-order ODE for a harmonic oscillator:
\begin{equation} \label{eqn:exp_growth} 
    \ddot{x} = -x
\end{equation}
substitution of the ansatz $x(t)=e^{\lambda t}$ into the ODE yields the characteristic equation:
\begin{equation} \label{eqn:exp_growth} 
    \lambda^2 = -1
\end{equation}
which can be solved for the roots:
\begin{equation} \label{eqn:exp_growth} 
    \lambda = \pm i
\end{equation}




\subsection{The importance of the characteristic polynomial}

Note how the characteristic polynomial is derived from substitution of the ansatz $x(t)=e^{\lambda t}$ into the ODE, and the resulting roots can be real or complex.

\section{Decomposition of Higher-Order ODEs into a Set of Coupled First-Order ODEs}


Consider again the ODE for a harmonic oscillator:
\begin{equation} \label{eqn:exp_growth2} 
    \ddot{x} = -x
\end{equation}
Systematically assign a different indexed variable for each derivative of $x$ including the zeroth order:
\begin{equation}
    x_1 = x 
\end{equation}
\begin{equation} \label{eqn:x_2}
    x_2 = \dot{x_1}
\end{equation}
The original ODE \eqref{eqn:exp_growth} can thus be written:
\begin{equation} \label{eqn:dot_x_2}
    \dot{x_2} = -x_1
\end{equation}
Equations \eqref{eqn:x_2} and \eqref{eqn:dot_x_2} can be written as a coupled linear system of equations:
\begin{equation} 
\frac{d}{dt} \begin{bmatrix} x_1 \\ x_2 \end{bmatrix} = 
\begin{bmatrix}
0 & 1 \\ -1 & 0
\end{bmatrix}
\begin{bmatrix}
x_1 \\ x_2
\end{bmatrix}
\label{eqn:coupled_linear_harmonic_osc_ode}
\end{equation}
This equation has the general form:
\begin{equation} 
\frac{ d \bf{x} }{dt}  = 
\bf{A} \bf{x}
\label{eqn:general_coupled_linear_odes}
\end{equation}
Note that a decoupled system will have a matrix $\bf{A}$ this is diagonal.  
\end{document}